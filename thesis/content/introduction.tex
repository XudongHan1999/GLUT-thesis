\section{引言}
\subsection{研究背景}
激光器和超低损耗光纤的问世不仅引起了光纤通信的革命,而且拉开了非线性光学的序幕。非线性光学是现代光学很重要的分支,它主要研究强光(主要是激光)与物质相互作用产生的各种非线性效应,比如 Kerr 效应、受激散射效应、多波混频效应和光孤子效应等。1973 年,Hasegawa 和 Tappert 在研究脉冲在光纤中传输的课题时认为,反常群速度色散和非线性效应的相互作用,可以使孤子脉冲在低损光纤中传输,且其传输过程由非线性 Schr\"odinger 方程描述。他们还指出,若在通信系统用光孤子作为载体,可使通信容量比线性通信系统提高 1$\sim$2 个数量级\cite{Hasegawa}。光孤子通信的广阔前景引起人们的兴趣,对它进行了广泛的研究。

理论上对光孤子的研究就是求解非线性 Schr\"odinger 方程。1971 年,Zakharov 和 Shabat 用逆散射方法对非线性 Schr\"odinger 方程进行求解,得到该方程的解析解。然而,逆散射方法求解非线性 Schr\"odinger 方程需要求解复杂的积分方程,因此数值模拟成为研究非线性 Schr\"odinger 方程(光孤子问题)必要的一种手段。除了求解偏微分方程通用的有限差分法外,Hardin 和 Tappert 在 1972 年提出用 分步 Fourier 方法求解非线性偏微分方程\cite{Tappert};1978 年,Fornberg 和 Whitham 提出用伪谱方法求解非线性偏微分方程\cite{Fornberg}。非线性 Schr\"odinger 方程的数值求解方法有很多,Taha 和 Ablowitz 在 1983 年对非线性 Schr\"odinger 方程的数值方法进行了总结和比较\cite{Taha}。

\subsection{本文工作}
本文的主要工作是在 MATLAB 上应用有限差分、分步 Fourier 方法和伪谱方法这三种数值方法实现非线性 Schr\"odinger 方程的数值求解。求解目标是得到现单光孤子周期性演化的物理图像。在本文的第\ref{sec:NLSE}章,从光纤介质的本构关系出发出发推导了非线性 Schr\"odinger 方程;在本文的第\ref{sec:optical soliton}章,介绍了孤子理论,着重介绍了但光孤子的传输;在本文的第\ref{sec:numerical}章,阐述了用三种数值方法对光孤子效应进行数值模拟的工作;在本文的最后部分,对三种数值方法进行了总结和对比。